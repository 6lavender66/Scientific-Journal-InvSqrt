\documentclass[12pt]{article}
\usepackage[polish]{babel} 
\usepackage[T1]{fontenc} % obsługa polskich znaki
\usepackage[utf8]{inputenc} % użycie kodowania UTF-8 dla pliku źródłowego
\usepackage{enumitem} % dodatkowe narzędzie do konfiguracji list

\begin{document}

Rozplanowanie artykułu (zawiera podrozdziały jakie powinien takowy zawierać + rzeczy które mam napisać z wiadomości prywatnej i zadania

\begin{enumerate}
    \item Tytuł, autor etc
    \item Streszczenie (Abstract)
    \item Wprowadzenie (Introduction)
    \begin{itemize}
        \item Omówienie problemu - czyli jaki jest problem i jak ten algorytm go rozwiązuje
    \end{itemize}
    \item Materiały (Materials) 
    \begin{itemize}
        \item Rzeczy z których korzystałem
    \end{itemize}
    \item Metody (Methods)
    \begin{itemize}
        \item Przykładowa implementacja w Pythonie albo w C(sharp)
        \item Omówienie kilku algorytmów związanych z tym tematem
        \item Kalkulację złożoności obliczeniowej
        \item Jak został wykorzystany w Quake'u (z tego, co pamiętam to w III dopiero...)
        \item Dlaczego implementacje w tych językach nigdy nie będą tak dobre jak ta w Quake'u?
    \end{itemize}
    \item Wyniki i dyskusja (Results and discussion)
    \begin{itemize}
        \item Do czego jest wykorzystywany dzisiaj i do czego się przyczynił
    \end{itemize}
    \item Wnioski (Conclusions)
    \begin{itemize}
        \item Wnioski blabla
    \end{itemize}
    \item Podziękowania (Acknowledgments)
    \begin{itemize}
        \item (opcjonalne podziękowania, odrazu wpisz SciHub)
    \end{itemize}
    \item Bibliografia (References)
    \begin{itemize}
        \item Referencje, cytowania etc
    \end{itemize}
\end{enumerate}
\end{document}
